% !TEX TS-program = xelatex
% !TEX encoding = UTF-8 Unicode
% !Mode:: "TeX:UTF-8"

\documentclass{resume}
\usepackage{zh_CN-Adobefonts_external} % Simplified Chinese Support using external fonts (./fonts/zh_CN-Adobe/)
%\usepackage{zh_CN-Adobefonts_internal} % Simplified Chinese Support using system fonts
\usepackage{linespacing_fix} % disable extra space before next section
\usepackage{cite}

\begin{document}
\pagenumbering{gobble} % suppress displaying page number

\name{倪思宇}

% {E-mail}{mobilephone}{homepage}
% be careful of _ in emaill address
\contactInfo{nisiyu91@sina.com}{(+86) 131-228-73517}{}
% {E-mail}{mobilephone}
% keep the last empty braces!
%\contactInfo{xxx@yuanbin.me}{(+86) 131-221-87xxx}{}

\section{\faGraduationCap\  教育背景}
\datedsubsection{\textbf{上海交通大学}, 上海}{2009 年9 月 -- 2013 年6 月}
\textit{学士}\ 计算机科学与技术专业

\section{\faUsers\ 工作经历}
\datedsubsection{\textbf{华为技术有限公司} 上海}{2013 年8 月 -- 至今}
\role{全职} {安全工程师}
\begin{itemize}
  \item 负责无线产品线超过5个版本的通信设备安全测试评估
  \begin{itemize}
    \item 结合STRIDE威胁分析建模的方法进行安全测试设计制定测试策略,测试执行和报告。
    \item web安全 熟练使用代理工具Burp以及浏览器开发者工具做SQLi,XSS, CSRF 相关测试,熟悉HTTP 协议,了解web的基本前后端技术
    \item FUZZ测试 熟练使用Peach Fuzzer框架,编写数据模型、状态模型、监控器进行模糊测试
    \item C代码审计 熟练运用商用工具如Coverity,Fortify进行代码审计,了解CWE中常见的软件缺陷类型,了解llvm中自带的代码扫描工具
    \item Python脚本 熟练使用python进行一些测试脚本/POC的开发,了解scapy、sqlmap、paramiko、mitmf等工具来帮助提升测试效率
    \item 根据业界规范、公司标准进行操作系统、数据库、web容器的配置检查
    \item 对x86平台 PE文件逆向工程和调试,llvm中klee符号执行等代码分析方式略有了解
  \end{itemize}
  
  \item 负责产品认证项目的对外支撑
  \begin{itemize}
    \item 了解Common Criteria标准的大致流程
    \item 为产品编写认证用测试用例
    \item 独自前往西班牙,承担设备调试,与认证机构人员全英文交流认证项目
  \end{itemize}
  
  \item 小工具开发
  \begin{itemize}
    \item C代码历史问题排查工具 使用python开发的轻量级脚本,使用pylex的语法分析技术,进行一些敏感特征的正则匹配以及做一些上下文的搜索,将部分问题举一反三
    \item 例行事项提醒Chrome插件 javascript开发,通过轮询查看例行事项有无完成,来改变图标颜色,使用浏览器时就可时刻注意到
    \item 在线代码扫描webapp 为第三方命令行扫描工具做页面包装,使用Django框架搭建webapp,解析工具XML结果文件转化为页面表格,使用Bootstrap前端框架呈现
  \end{itemize}
\end{itemize}



% Reference Test
%\datedsubsection{\textbf{Paper Title\cite{zaharia2012resilient}}}{May. 2015}
%An xxx optimized for xxx\cite{verma2015large}
%\begin{itemize}
%  \item main contribution
%\end{itemize}

\section{\faCogs\ IT 技能}
% increase linespacing [parsep=0.5ex]
\begin{itemize}[parsep=0.5ex]
  \item 编程语言: 
  \begin{itemize}
  \item 熟练:C,Python 
  \item 一般:C++,Javascript, PHP,shell
  \end{itemize}
  \item web开发: 
  \begin{itemize}
  \item 后端:Django,Yii Framework
  \item 前端:Bootstrap, jQuery,Chrome插件
  \end{itemize}
  \item 版本控制:Git svn
\end{itemize}


\section{\faInfo\ 其他}
% increase linespacing [parsep=0.5ex]
\begin{itemize}[parsep=0.5ex]
  \item 个人主页: http://www.nisiyu.com
  \item GitHub: https://github.com/nisiyu91
  \item 语言: 英语 - 技术文档阅读无压力,英文email交流无压力 
\end{itemize}

%% Reference
%\newpage
%\bibliographystyle{IEEETran}
%\bibliography{mycite}
\end{document}
